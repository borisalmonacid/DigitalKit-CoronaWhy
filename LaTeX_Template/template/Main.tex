\documentclass[10pt]{definitions/template}


\Title{Article Template for Corona Why}

% Create a new command for each orcid.
% To create a new command, the following command must be used:
% \newcommand{\orcidauthorXY}{0000-0000-000-0000}
% Where X and Y is a letter between A and Z. Altogether the value for X and Y must be different for each author, since it corresponds to an identifier.

% Example:

% Author 1, uses the identifier AA.
\newcommand{\orcidauthorAA}{0000-0000-000-0000}
% Author 2, uses the identifier AB.
\newcommand{\orcidauthorAB}{0000-0000-000-0000}

% To be able to call this command, you must use the script \orcidXY{}


% Authors, for the paper.
\Author{
Firstname Lastname $^{1}$ \orcidAA{},
Firstname Lastname $^{1, 2}$,
Firstname Lastname $^{1}$ \textsuperscript{\faEnvelopeO} \orcidAB{},
and
Firstname Lastname $^{2}$
}


% Affiliations/Addresses.
\address{%
\begin{packed_enum}
    \item[1] Affiliation 1; e-mail@e-mail.com
    \item[2] Affiliation 2; e-mail@e-mail.com
    \item[3] Affiliation 3; e-mail@e-mail.com
    \item[\textsuperscript{\faEnvelopeO}] Corresponding author: info@coronawhy.org
\end{packed_enum}
}

% One example and tutorial can be found in http://www.cbs.umn.edu/sites/default/files/public/downloads/Annotated_Nature_abstract.pdf
\abstract{
A single paragraph of about 200 or 300 words maximum. For research articles, abstracts should give a pertinent overview of the work. 
}

% Keywords
\keyword{
keyword 1;
keyword 2;
keyword 3;
keyword n.
}

\dataset{
Add the corresponding link to the DOI or a link to the dataset that is in a repository (github, gitlab). To get a DOI number for github, please read the following article \url{https://guides.github.com/activities/citable-code/}
}

\datasetlicense{
License under which the data set is made available (CC0, CC-BY, CC-BY-SA, CC-BY-NC, etc.)
}

\keycontribution{
The breakthroughs or highlights of the manuscript. Authors can write one, two, three or four sentences to describe the most important part of the paper.
}


% Graphical abstract (Optional)
% If you want to change the figure, you must overwrite the graphical_abstract.png file in the figure folder.
% If you do not want to add a graphic abstract, you should comment on this command.
\graphicalabstract{
A sentence that briefly explains the graphical abstract.
}

\linenumbers % Remove for publish.

\begin{document}

\maketitle % print the title


\newpage
\section{Introduction}

\blindtext %remove
\par\null %remove
\blindtext %remove
\par\null %remove
\blindtext %remove

\section{Materials and Methods}

\blindtext % remove


\subsection{Subsection}

\blindtext %remove

\subsubsection{Subsubsection}

Bulleted lists:
\begin{itemize}[leftmargin=*,labelsep=5.8mm]
\item	First bullet
\item	Second bullet
\item	Third bullet
\end{itemize}

Numbered lists:
\begin{enumerate}[leftmargin=*,labelsep=4.9mm]
\item	First item 
\item	Second item
\item	Third item
\end{enumerate}

\blindtext %remove

\subsection{Equation}

The mass-energy equivalence is described by the famous Equation \ref{Equation:emc2} discovered in 1905 by Albert Einstein.

\begin{equation}
\label{Equation:emc2}
E=mc^2
\end{equation}

 
More information can be found in \url{https://www.overleaf.com/learn/latex/mathematical_expressions}. A list of the Greek letters and math symbols can be found in \url{https://www.overleaf.com/learn/latex/List_of_Greek_letters_and_math_symbols}.

\subsection{Matrices}

Matrix with square brackets is shown in the Equation \ref{Equation:brackets_matrix}. Other examples of matrices can be found in \url{https://www.overleaf.com/learn/latex/Matrices}.

\begin{equation}
\label{Equation:brackets_matrix}
\begin{bmatrix}
1 & 2 & 3\\
a & b & c
\end{bmatrix}
\end{equation}


\subsection{Chemistry Formulae}

The template include the chemfig package, more information about this package can be found in \url{https://www.overleaf.com/learn/latex/Chemistry_formulae}. The branched molecule is describe in Figure \ref{Figure:branched_molecule}.


\begin{figure}[H]
\label{Figure:branched_molecule}
\centering
\chemfig{H-C(-[2]H)(-[6]H)-C(=[1]O)-[7]H}
\caption{Branched molecule.}
\end{figure}






\subsection{Algorithm}

If you need information about this package, please see \url{http://ctan.dcc.uchile.cl/macros/latex/contrib/algorithm2e/doc/algorithm2e.pdf}. Algorithms should be placed in the main text near to the first time they are cited. The algorithm \ref{algorithm:example} is cited here.

\begin{figure}[htb]

\centering
\begin{minipage}{0.7\linewidth}
 % Start algorithm.
\begin{algorithm}[H]
\label{algorithm:example}
    \setstretch{1.0}
    \DontPrintSemicolon
      
    \KwInput{Your Input}
    \KwOutput{Your output}
    \KwData{Testing set $x$}
    %-----
    $\sum_{i=1}^{\infty} := 0$ \tcp*{this is a comment.}
    \tcc{Now this is an if...else conditional loop.}
    \If{Condition 1}
    {
        Do something    \tcp*{this is another comment.}
        \If{sub-Condition}
        {
            Do a lot
        }
    }
    \ElseIf{Condition 2}
    {
        Do Otherwise \;
        \tcc{Now this is a for loop.}
        \For{sequence}    
        { 
        	loop instructions \;
        }
    }
    \Else
    {
    	Do the rest
    }
        
    \tcc{Now this is a While loop.}
    \While{Condition}
    {
        Do something\;
    }
\caption{Example code}
\end{algorithm}
%----- End algorithm
\end{minipage}
\end{figure}

\subsection{How Cite?}

In this form \cite{adams1995hitchhiker}.

\section{Results}

\blindtext %remove

\subsection{Figures, Tables and Schemes}

All figures and tables should be cited in the main text as Figure \ref{figure:example_figure},Table \ref{table:table}, Table \ref{table:big_table}, etc.

\begin{figure}[H]
\label{figure:example_figure}
\centering
\includegraphics[width=10cm]{figures/graphical_abstract.png}
\caption{This is a figure. Figures should be placed in the main text near to the first time they are cited.}
\end{figure}   
 

\begin{table}[H]
\label{table:table}
\caption{This is a table caption. Tables should be placed in the main text near to the first time they are cited.}
\centering
\begin{tabular}{ccc}
\toprule
\textbf{Title 1}	& \textbf{Title 2}	& \textbf{Title 3}\\
\midrule
entry 1		& data			& data\\
entry 2		& data			& data\\
\bottomrule
\end{tabular}
\end{table}



\begin{table}[H]
\label{table:big_table}
\caption{This is a big table}
\centering
% \tablesize{} %% You can specify the fontsize here, e.g., \tablesize{\footnotesize}. If commented out \small will be used.
% Use this command \tablesize{\footnotesize} only when there is no different way to display the information.. Please prefer to divide the table in two or manage the information in sets of tables, remember that there are people who cannot read very small letters.
\begin{tabular}{ccccccccccc}
\toprule
\textbf{Title 1} & \textbf{Title 2}	& \textbf{Title 3} & \textbf{Title 4} & \textbf{Title 5}	& \textbf{Title 6} & \textbf{Title 7} & \textbf{Title 8} & \textbf{Title 9} & \textbf{Title 10}\\
\midrule
entry 1		& data & data & data & data & data & data & data & data & data \\
entry 2		& data & data & data & data & data & data & data & data & data \\
entry 3		& data & data & data & data & data & data & data & data & data \\
entry 4		& data & data & data & data & data & data & data & data & data \\
entry 5		& data & data & data & data & data & data & data & data & data \\
entry 6		& data & data & data & data & data & data & data & data & data \\
entry 7		& data & data & data & data & data & data & data & data & data \\
entry 8		& data & data & data & data & data & data & data & data & data \\
entry 9		& data & data & data & data & data & data & data & data & data \\
\bottomrule
\end{tabular}
\end{table}


\input{Discussion}

\input{Conclusions}

\acknowledgments{
In this section you can acknowledge any support given which is not covered by the author contribution or funding sections.
}

\funding{
Example 1: ``This research received no external funding''. Example 2: ``This research was funded by ABC.'' and  and ``The APC was funded by ABC''.
}

\conflictsofinterest{
Indicate if there is a conflict of interest, in the event that there is no conflict of interest, leave the following sentence: ``The authors declare no conflict of interest.''.
} 

\authorcontributions{
Example: The following statements should be used ``Conceptualisation, X.X. and Y.Y.; methodology, X.X.; software, X.X.; validation, X.X., Y.Y. and Z.Z.; formal analysis, X.X.; investigation, X.X.; resources, X.X.; data curation, X.X.; writing--original draft preparation, X.X.; writing--review and editing, X.X.; visualisation, X.X.; supervision, X.X.; project administration, X.X.; funding acquisition, Y.Y.; etc.
Review the positions or multiple roles that may exist.
}

\dedication{
A formal message.
}

\newpage
\bibliographystyle{plain}
\bibliography{references/references}


\newpage

\appendix
\section*{Supplementary Material}
\addcontentsline{toc}{section}{Appendices}
\renewcommand{\thesubsection}{\Alph{subsection}}

%===== Appendix Subsections
\subsection{Appendix Subsection}

\blindtext

\subsection{Another appendix Subsection}

\blindtext



\end{document}
